% !TeX TS-program = xelatex

\documentclass{beamer}

\usetheme{metropolis}

\usepackage[utf8]{inputenc}
\usepackage{amsthm, amssymb, mathtools, amsmath, bbm, mathrsfs, stmaryrd, xcolor}
\usepackage{nicefrac}
\usepackage[parfill]{parskip}
\usepackage{float}
\usepackage{cleveref}
\usepackage{svg}
\usepackage{tikz-cd}
\usepackage{multicol}
\usepackage{listings}

\usepackage{color}
\definecolor{keywordcolor}{rgb}{0.7, 0.1, 0.1}   % red
\definecolor{tacticcolor}{rgb}{0.0, 0.1, 0.6}    % blue
\definecolor{commentcolor}{rgb}{0.4, 0.4, 0.4}   % grey
\definecolor{symbolcolor}{rgb}{0.0, 0.1, 0.6}    % blue
\definecolor{sortcolor}{rgb}{0.1, 0.5, 0.1}      % green
\definecolor{attributecolor}{rgb}{0.7, 0.1, 0.1} % red

\def\lstlanguagefiles{lstlean.tex}
% set default language
\lstset{language=lean}

\newcommand{\N}{\mathbb{N}}
\newcommand{\Z}{\mathbb{Z}}
\newcommand{\NZ}{\mathbb{N}_{0}}
\newcommand{\Q}{\mathbb{Q}}
\newcommand{\R}{\mathbb{R}}
\newcommand{\C}{\mathbb{C}}
\newcommand{\A}{\mathbb{A}}
\newcommand{\proj}{\mathbb{P}}
\newcommand{\sheaf}{\mathcal{O}}
\newcommand{\FF}{\mathcal{F}}
\newcommand{\G}{\mathcal{G}}

\newcommand{\zproj}{Z_{\textnormal{proj}}}

\newcommand{\maxid}{\mathfrak{m}}
\newcommand{\primeid}{\mathfrak{p}}
\newcommand{\primeidd}{\mathfrak{q}}

\newcommand{\F}{\mathbb{F}}
\newcommand{\incl}{\imath}

\newcommand{\tuple}[2]{\left\langle#1\colon #2\right\rangle}

\DeclareMathOperator{\order}{order}
\DeclareMathOperator{\Id}{Id}
\DeclareMathOperator{\im}{im}
\DeclareMathOperator{\ggd}{ggd}
\DeclareMathOperator{\kgv}{kgv}
\DeclareMathOperator{\degree}{gr}
\DeclareMathOperator{\coker}{coker}
\DeclareMathOperator{\matrices}{Mat}

\DeclareMathOperator{\gl}{GL}

\DeclareMathOperator{\Aut}{Aut}
\DeclareMathOperator{\Hom}{Hom}
\DeclareMathOperator{\End}{End}
\DeclareMathOperator{\colim}{colim}
\newcommand{\isom}{\overset{\sim}{\longrightarrow}}

\newcommand{\schemes}{{\bf Sch}}
\newcommand{\aff}{{\bf Aff}}
\newcommand{\Grp}{{\bf Grp}}
\newcommand{\Ab}{{\bf Ab}}
\newcommand{\cring}{{\bf CRing}}
\DeclareMathOperator{\modules}{{\bf Mod}}
\newcommand{\catset}{{\bf Set}}
\newcommand{\cat}{\mathcal{C}}
\newcommand{\chains}{{\bf Ch}}
\newcommand{\homot}{{\bf Ho}}
\DeclareMathOperator{\objects}{Ob}
\newcommand{\gen}[1]{\left<#1\right>}
\DeclareMathOperator{\cone}{Cone}
\newcommand{\set}[1]{\left\{#1\right\}}
\newcommand{\setwith}[2]{\left\{#1:#2\right\}}
\DeclareMathOperator{\Ext}{Ext}
\DeclareMathOperator{\Nil}{Nil}
\DeclareMathOperator{\idem}{Idem}
\DeclareMathOperator{\rad}{Rad}
\DeclareMathOperator{\divisor}{div}
\DeclareMathOperator{\Pic}{Pic}
\DeclareMathOperator{\spec}{Spec}
\DeclareMathOperator{\supp}{Supp}
\newcommand{\ideal}{\triangleleft}
\newcommand{\ff}{\mathcal{K}}

\newtheorem{proposition}{Proposition}

\title{Closed immersions of schemes in Lean}
\author{Amelia Livingston \and Torger Olson \and Wim Nijgh \and Jonas van der Schaaf}
\date{July 14, 2023}

\begin{document}
\maketitle

\section{Why is algebraic geometry hard?}

\begin{frame}
    \frametitle{Spectra of rings}

    \begin{itemize}
        \item Given a commutative ring \(R\) we define
              \[
                  \spec R=\setwith{\primeid\ideal R}{\textnormal{\(\primeid\) prime}}
              \]
        \item This can be given a topology with the closed sets given by:
              \[
                  V(I)=\setwith{I\subseteq \primeid}{\primeid \in \spec R},
              \]
        where $I\ideal R$ is an ideal. 
        \item We can define ``functions'' (sheaf) on each basic open \(D_{f}=\spec
              R\setminus V(f)\)
              \[
                  \Gamma(D_{f},\sheaf_{\spec R})=R\left[\frac{1}{f}\right]
              \]
        \item A scheme is a topological space with functions on each open, which
              locally looks like $\spec R$
    \end{itemize}

\end{frame}

\begin{frame}
    \frametitle{Closed immersions}

    \begin{itemize}
        \item We have a correspondence:
              \[
                  \spec R/I\simeq\setwith{\primeid\in\spec R}{I\subseteq\primeid}\subseteq\spec R
              \]

        \item This induces a closed embedding of topological spaces

        \item It is surjective on the functions: \(R\twoheadrightarrow R/I\) is
              surjective

        \item Morphisms with this property are called closed immersions
    \end{itemize}

\end{frame}

\section{Why is Lean hard?}

\begin{frame}
    \frametitle{Schemes in Lean}

    \begin{itemize}
        \item Absolutely no handwaving

        \item So many different structures (including morphisms of them and
              functors between the categories):

              \begin{multicols}{2}
                  \begin{enumerate}
                      \item CommRingCat
                      \item Modules
                      \item Scheme
                      \item LocallyRingedSpace
                      \item SheafedSpace
                      \item PresheafedSpace
                      \item AffineScheme
                      \item PrimeSpectrum
                      \item Topology
                      \item Stalks
                      \item MorphismProperty
                  \end{enumerate}
              \end{multicols}
        \item These are just what I could come up with on my way home
    \end{itemize}
\end{frame}



\end{document}